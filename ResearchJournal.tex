% Please do not change the document class
\documentclass{scrartcl}

% Please do not change these packages
\usepackage[hidelinks]{hyperref}
\usepackage[none]{hyphenat}
\usepackage{setspace}
\usepackage{graphicx}
\usepackage{subcaption}
\doublespace

% You may add additional packages here
\usepackage{amsmath}

% Please include a clear, concise, and descriptive title
\title{COMP210 - Research Journal}

% Please put your student number in the author field
\author{1507729}

\begin{document}

\maketitle

\section{Putting Man above Motion}

Virtual reality (VR) isn't a new idea, with some even claiming it's been around since the 1950s. \cite{vrs2017origin} There have been many attempts at getting VR into the public however each time the technology to support VR hasn't been able to keep up. While developments happen in VR technology it's popularity has grown within the gaming community however recent VR devices can cause users to feel ill for a number of reasons. \cite{porcino2017minimizing} The term used is called cyber sickness though it's also been called simulator sickness \cite{gower1989simulator} previously.

There have been many studies around how to deal with cyber sickness and even with knowledge of what effects it there is still the question of what interface users find to be the best to help dealing with the disparity between a head mounted display (HMD) some of these studies even go so far as to show rankings between various methods and how it affected users. \cite{benzeroual2013cyber, mentzelopoulos2015hardware} Of course changing how you interact with an avatar isn't just one of the variables you can alter. It's also possible to alter the user experience by experimenting with unique non-standard avatars \cite{won2015homuncular}, which as users begin to embody could help to immerse them into the behaviours which could help to reduce their cyber sickness for certain actions. This is largely due to what psychology call plasticity, saying that the human mind can adapt and change to it's circumstances. \cite{madary2016real}

\section{Realities Heuristics vs Virtual Realities Heuristics}

It's also worth mentioning that the definition of virtual reality can be questioned as to the true extent of which the technology and software need to reach before they can be considered as VR. One could consider a statement made by Lanier \cite{lanier1992virtual} Virtual reality was originally conceived as a digitally created space that humans could access by donning sophisticated computer equipment as an acting definiton for VR however this would mean that augmented reality (AR) is also VR,  Rosa et al. talks about exactly where the definition for AR should be made to distinguish between the core concept of VR and that of reality making a point of mentioning if you have a digitally captured and displayed element is real or virtual. \cite{rosa2016re} Though  Milgram et al. \cite{milgram1994taxonomy} show the virtuality continuum as a clear line from reality to virtual and the steps between. The need for a clear definition is obvious when it comes to creating a heuristic model and means that the model is fit for purpose rather than containing non-relevant requirements. 

For some the heuristic model proposed by the Nielsen \cite{nielsen1990heuristic} is the model to use, however it's been called into question by many over the years as to how fitting the model is for a variety of industries. \cite{pinelle2008heuristic, pinelle2009usability, sutcliffe2004heuristic, john1998traditional} The question of Human Computer Interaction (HCI) techniques being out-dated, is something that has to be addressed properly \cite{hvannberg2012exploitation}. Although the Nielsen model covers many of the requirements for users it was made many years ago, and although the core principles remain the same the ways in which technology interacts with people certainly couldn't have been predicted when the paper was originally written. 

One issue with not having a more concrete model means that there is there enough work in the field of HCI beyond that of debating the heuristic model which should be used. Glass et al. \cite{glass2004analysis} looked at papers within the fields of computer science, information systems and software engineering with only a collective 5.7\% of the over 1000 papers published across all three fields were to do with HCI. With people still publishing papers on the topic of HCI and how to enhance a users experience it's possible that this field hasn't been as fully explored and with video games and virtual reality being both relatively new and rapidly advancing fields establishing a strong foothold could make a huge difference in shaping future research and development within both industries.

\section{What in Reality Looks Best Virtuality}
While it's easy to assume how information could be displayed in both VR and within AR it's worth noting what information should be display. Ventä-Olkkonen et al \cite{venta2014investigating}. performed a rudimentary study, where they simulated a mixed reality application, which shines some insight into what a user initially believes they want likely based on a precursor study they carried out \cite{venta2012user}. This however doesn't consider if user preference changes after prolonged use Karapanos et al. \cite{karapanos2009user} created a framework around the idea of measuring a user experience after the initial experience.


\bibliographystyle{ieeetran}
\bibliography{references}

\end{document}
